\documentclass[main.tex]{subfiles}

\begin{document}
\chapter[Л/р~2 Линейный вычислительный процесс]{Лабораторная работа №~2. Линейный вычислительный процесс}

\textbf{Цели и задачи работы}: изучение функций ввода-вывода данных в С++, программирование вычисления значения выражения.

\textbf{Теоретические сведения о работе} приведены в литературе \cite{vs2010,qt2019}, конспекте лекций.
Примеры программ на С++ размещены в приложении \ref{ch:app02} (страница~\pageref{ch:app02}).

\textbf{Задание к работе}: реализовать линейный вычислительный процесс на С++.
Самостоятельно решить задачу в соответствии с индивидуальным вариантом.

\section{Индивидуальные варианты заданий: задание №~1}

\textit{Примечание: все входные и выходные данные в заданиях являются
вещественными числами.}
\\

% 
\begin{enumerate}
    \item Дан радиус окружности $R$, найти длину окружности $L$ и площадь круга $S$, ограниченного этой окружностью. $L = 2 \pi R$, $S = \pi R^2$.
    \item Дана сторона квадрата $a$. Найти квадрат его периметра $P = 4a$.
    \item Дана сторона квадрата $a$. Найти его площадь $S = a^2$.
    \item Даны стороны прямоугольника $a$ и $b$. Найти его площадь $S = a b$ и периметр $P = 2 (a + b)$.
    \item Дан диаметр окружности $d$. Найти её длину $L = 2 \pi R$.
    \item Дана длина ребра куба $a$. Найти объем куба $V = a^3$ и площадь его поверхности $S = 6 a^2$.
    \item Дана площадь круга $S$. Найти его диаметр $D$ и длину окружности $L$, ограничивающей этот круг. $L = 2 \pi R$, $S = \pi R^2$.
    \item Даны длины ребер $a$, $b$, $c$ прямоугольного параллелепипеда. Найти
    его объем $V = a b c$ и площадь поверхности $S = 2 (a b + b c + a c)$.
    \item Дана длина окружности $L$. Найти её радиус $R$ и площадь круга $S$, ограниченного этой окружностью. $L = 2 \pi R$, $S = \pi R^2$.
    \item Даны два числа $a$ и $b$. Найти их среднее арифметическое.
    \item Даны два круга с общим центром и радиусами $R_1$ и $R_2$ ($R_1 > R_2$). Найти площади этих кругов $S_1$ и $S_2$, а также площадь $S_3$ кольца, образованного этими концентрическими окружностями. $S = \pi R^2$.
    \item Дано значение угла $\alpha$ в градусах ($0 \leq \alpha < 360$). Найти значение угла в радианах.
    \item Вывести результат функции $y = x + 0.2$ при $x = 0.1$ с точностью 18 знаков после запятой.
    \item Дано значение угла $\alpha$ в радианах ($0 \leq \alpha < 2 \pi$). Найти значение угла в градусах.
    \item Даны два неотрицательных числа $a$ и $b$. Найти их среднее геометрическое $res = \sqrt{a b}$.
    \item Даны катеты прямоугольного треугольника $a$ и $b$. Найти его гипотенузу $c$ и периметр $P$.
    \item Даны две точки на числовой оси $x_1$, $x_2$. Найти расстояние между ними. $len = |x_2 - x_1|$.
    \item Даны три точки на числовой оси $a$, $b$, $c$, где $a < c < b$. Найти сумму длин отрезков $ac$ и $bc$.
    \item Даны три точки на числовой оси $a$, $b$, $c$, где $a < c < b$. Найти произведение длин отрезков $ac$ и $bc$.
    \item Даны три точки на числовой оси $a$, $b$, $c$, где $a < c < b$. Найти среднее арифметическое длин отрезков $ac$ и $bc$.
    \item Найти значение функции $y = 3 x^2 - 6 x - 7$ при $x = 42.2$.
    \item Дано число $x$. Найти значение функции $y = 6 x^2 + 3 x + 7$.
    \item Даны числа $x_1$, $x_2$. Найти значение функции $y = \sqrt{x_2^2 - x_1^2}$.
    \item Дана стоимость товара и размер скидки ($0.0 \leq discount\leq 1.0$). Вывести стоимость товара со скидкой.
    \item Дана стоимость товара и размер скидки в процентах (от 0~\% до 100~\%). Вывести стоимость товара со скидкой.
    \item Даны два ненулевых числа. Найти сумму и разность их квадратов.
    \item Даны два ненулевых числа. Найти сумму и разность их модулей.
    \item Даны два ненулевых числа. Найти произведение и частное их квадратов.
    \item Даны два ненулевых числа. Найти произведение и частное их модулей.
    \item Поменять местами содержимое переменных $value1$ и $value2$ и вывести новые значения $value1$ и $value2$.
\end{enumerate}


\section{Индивидуальные варианты заданий: задание №~2}

\textit{Примечание: все входные и выходные данные в заданиях являются целыми положительными числами больше нуля.}
\\

% divmod
\begin{enumerate}
    \item Дано расстояние в сантиметрах. Найти количество полных метров в нём.% (1~м. = 100~см.).
    \item Дана масса в килограммах. Найти количество полных тонн в ней.% (1~т. = 1000~кг.).
    \item Дан размер файла в байтах. Найти количество полных кибибайт (КиБ), которые занимает файл (1~КиБ = 1024~байта).
    \item Дана длительность в минутах. Найти количество полных часов.% (1~ч. = 60~мин.).
    \item Дана длительность в месяцах. Найти количество полных лет.% (1~год = 12~месяцев).
    \item Дана сумма в копейках. Вывести число, отбросив единицы копеек (пример: 1200045 -> "12000 руб. 40 коп.").
    \item Дано двузначное число. Найти сумму и произведение его цифр.
    \item Дано двузначное число. Вывести число, полученное при перестановке цифр исходного числа (пример: 12 -> 21).
    \item Дано трехзначное число. Найти сумму и произведение его цифр.
    \item Дано трехзначное число. Вывести число, полученное при прочтении исходного числа справа налево (пример: 123 -> 321).
    \item Дано трехзначное число. Получить двухзначное число, которое содержит вначале его последнюю цифру, а затем~--- его среднюю цифру (пример: 123 -> 32).
    \item Дано трехзначное число. Вывести первую цифру числа.
    \item Дана длительность в секундах. Вывести количество часов, минут и секунд (пример: 4344 -> "1:12:24").
    \item Дано трехзначное число. Вывести последнюю цифру числа.
    \item Дано трехзначное число. Занулить сотни в числе и вывести результат (пример: 123 -> 23).
    \item Дано трехзначное число. Занулить единицы и десятки в числе и вывести результат (пример: 123 -> 100).
    \item Дано трехзначное число. Занулить десятки в числе и вывести результат (пример: 123 -> 103).
    \item Дано трехзначное число. Занулить единицы в числе и вывести результат (пример: 123 -> 120).
    \item Дано трехзначное число. Вывести число, полученное при перестановке цифр сотен и десятков исходного числа (пример: 123 -> 213).
    \item Дано трехзначное число. Вывести число, полученное при перестановке цифр десятков и единиц исходного числа (пример: 123 -> 132).
    \item Дано количество студентов и количество яблок. Найти сколько яблок останется, если выдать всем студентам по одинаковому количеству яблок.
    \item Дана площадь пола (в м$^2$) и площадь одной плитки для пола (в м$^2$). Найти количество плиток, которое потребуется, чтобы замостить всю поверхность пола.
    \item Дано количество миллиметров $len1$ и сантиметров $len2$. Вывести число, равное сумме $len1 + len2$ в сантиметрах.
    \item Дано количество граммов $mass1$ и килограммов $mass1$. Вывести число, равное сумме $mass1 + mass2$ в граммах.
    \item Дано четырёхзначное число. Найти сумму его цифр.
    \item Дано четырёхзначное число. Найти произведение его цифр.
    \item Дано шестизначное число. Найти сумму его цифр.
    \item Дано шестизначное число. Найти произведение его цифр.
    \item Дана длительность в секундах. Найти количество полных минут.
    \item Дана длительность в секундах. Найти количество полных часов.
\end{enumerate}

\end{document}
