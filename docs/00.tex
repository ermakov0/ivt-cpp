\documentclass[main.tex]{subfiles}

\begin{document}
\newpage
%\chapter*{Введение}
%\addcontentsline{toc}{chapter}{Введение}

% Замечание для студентов заочной формы обучения: Студенты заочной
% формы обучения на занятиях по лабораторным работам выполняют только
% задания No1 лабораторных работ No1-8.

\section*{Порядок выполнения лабораторных работ}
\addcontentsline{toc}{section}{Порядок выполнения лабораторных работ}

\begin{enumerate}
    \item Познакомиться с темой и целью лабораторной работы;
    \item Изучить необходимые к выполнению лабораторной работы теоретические сведения и примеры программ;
    \item Познакомиться с общим заданием к лабораторной работе и индивидуальным вариантом задания;
    \item Написать и отладить программу решения задачи индивидуального варианта;
    \item Протестировать работу программы на различных наборах исходных данных;
    \item Продемонстрировать преподавателю работу программы;
    \item Оформить отчет;
    \item Защитить лабораторную работу.
\end{enumerate}

\section*{Оборудование, технические средства, инструмент}

Лабораторные работы выполняются в компьютерном классе, оснащенном персональными компьютерами.
На компьютерах должны быть установлены системы программирования \texttt{Microsoft Visual Studio} и \texttt{Qt Creator}.

\section*{Требования к оформлению отчета}
\addcontentsline{toc}{section}{Требования к оформлению отчета}

Отчет по лабораторной работе должен содержать:

\begin{itemize}
    \item титульный лист;
    \item цель работы;
    \item задание к лабораторной работе (общее задание и индивидуальный вариант задания);
    \item блок-схему алгоритма решения задачи;
    \item исходный код программы на С++;
    \item тестовые примеры, иллюстрирующие все варианты работы программы.
\end{itemize}

\end{document}
